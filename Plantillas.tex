
% Plantilla para agregar un glosario basico-
\newpage
\chapter*{Glosario}
\addcontentsline{toc}{chapter}{\protect Glosario}
\input{glosario}
% --------------------------------------------------------------------------------------------------

% Plantilla de ecuacion 
\begin{equation}\label{eq:Label}
	\bigtriangleup Rt = R_{o}\alpha\bigtriangleup t 
\end{equation}
% --------------------------------------------------------------------------------------------------

% Plantilla de figura 

\begin{figure}[tb]
	\centering
	\includegraphics[width=\textwidth]{letraA.png} % el archivo debe estar en la carpeta imagenes
	\caption[Titulo de la figura, este texto saldra en la lista de figuras]{\textbf{Titulo en negrita}. Texto descriptivo para la figura} 
	\label{fig:Agrande}
	% [Texto para la lista de figuras]{Texto para el caption}
\end{figure}
% --------------------------------------------------------------------------------------------------

% Plantilla para subfigure
\begin{figure}[tb]
	\centering
	\begin{subfigure}[t]{0.49\textwidth}
		\centering
		\includegraphics[width=\textwidth]{letraA.png} % el archivo debe estar en la carpeta imagenes
		\caption{\textbf{Texto para el caption a}}
		\label{fig:A}
	\end{subfigure}
	\hfill
	\begin{subfigure}[t]{0.49\textwidth}
		\centering
		\includegraphics[width=\textwidth]{letraB.png} % el archivo debe estar en la carpeta imagenes
		\caption{\textbf{Texto para el caption b}}
		\label{fig:B}
	\end{subfigure}
	
	\caption[Texto para la lista de figuras 1]{\textbf{Titulo en negrita}. Texto para el caption general \label{fig:Label}}
\end{figure}
% --------------------------------------------------------------------------------------------------

% Plantilla para agregar PDF's 
\includepdf[pages={33,19}, % Paginas a agregar {a - b} agrega las paginas de a hasta b
			landscape=true,
			fitpaper,
		%	templatesize={width}{height},
			templatesize={\paperwidth}{\paperheight},
			addtotoc={	33, subsubsection, 3, Indicador de presion, fig:Pindicador, % numero de pagina, seccion,nivel de seccion, texto, label
						19, subsubsection, 3, Accesorios para manómetros, fig:Pacc},
			addtolist={	33, figure, Catalogo de Indicador de presion, fig:Pindicador, % numero de pagina, tipo, texto, label
						19, figure, Catalogo de Accesorios para manómetros, fig:Pacc}]{directorio/"nombre de archivo pdf"}
% --------------------------------------------------------------------------------------------------

% Plantilla para tabla 
\begin{table}[tb]
	\centering
	\begin{threeparttable}
		\renewcommand{\arraystretch}{1.5} 	% Separacion de las filas
		\setlength{\tabcolsep}{6pt}			% Separacion de columnas
		\caption[Texto para la lista de tablas]{Texto del caption}
		\begin{tabular}{m{4em}m{5em}m{5em}m{7em}m{4em}m{6em}} % m, p y b para texto - c,l y r para numeros y autoexpansion
			\toprule
			 &  &  &  &  &  \\ \midrule
			 &  &  &  &  &  \\
			 &  &  &  &  &  \\ \bottomrule
		\end{tabular}
		\label{tab:Label}
		\begin{tablenotes}[flushleft]
			\item \vspace{7pt}\textbf{Nota.} Texto de la nota de tabla
		\end{tablenotes}
	\end{threeparttable}
\end{table}

% --------------------------------------------------------------------------------------------------

% Plantilla para tabla ajustada al texto
\begin{table}[tb]
	\centering
	\begin{threeparttable}
		\renewcommand{\arraystretch}{1.5} 	% Separacion de las filas
		%\setlength{\tabcolsep}{6pt}			% Separacion de columnas
		\caption[Texto para la lista de tablas]{Texto del caption, el Titulo}
		\begin{tabular*}{\textwidth}{c @{\extracolsep{\fill}} cccccc}
			\toprule
			numeracion & Dato 1 & Dato 2 & Dato 3 & Dato 4 & Dato 5 & Dato 6 \\ \midrule
			1      & 10     &   12   &   14   &   16   &   18   & 30     \\
			2      & 11     &   13   &   15   &   17   &   19   & 31     \\
			3      & 20     &   21   &   22   &   23   &   24   & 32     \\
			4      & 25     &   26   &   27   &   28   &   29   & 33     \\ \bottomrule
		\end{tabular*}
		\label{tab:Label1}
		\begin{tablenotes}[flushleft]
			\item \textbf{Nota.} Texto de la nota de tabla, descripcion de la tabla
		\end{tablenotes}
	\end{threeparttable}
\end{table}

% --------------------------------------------------------------------------------------------------

% Plantilla de sortlist (lista ordenada alfabeticamente)
\begin{sortedlist} 
	\sortitem{ \textbf{Flujo:} Movimiento de un fluido.}
	\sortitem{ \textbf{Flujo:} Movimiento de un fluido.}
\end{sortedlist}
% --------------------------------------------------------------------------------------------------

% Plantilla de footnote
\footnote[1]{} % marcador
\footnotetext[1]{Error combinado del sensor (histéresis, linealidad y repetibilidad)} % texto
% --------------------------------------------------------------------------------------------------

% Plantilla de minted (anexar codigo)
\inputminted[	frame=lines,
				obeytabs=true,
				tabsize=4,
				framesep=5mm,
				bgcolor=white,
				baselinestretch=1.5,
				fontsize=\tiny,]{python}{Ganancias_fuzzy.py}


\begin{listing}[H]
	\captionof{listing}{Follow behavior}
	\label{code:1}				
	\begin{minted}[escapeinside=||,mathescape=true, autogobble=true]{python}
	Codigo.... 
	\end{minted}
\end{listing}

\begin{minted}[xleftmargin=0.5cm, escapeinside=||,mathescape=true, 						autogobble=true,
obeytabs=true,
tabsize=4,
framesep=5mm,
bgcolor=Snow1,
baselinestretch=1.5,
fontsize=\footnotesize]{python}
	Codigo.... 
\end{minted}

\blockquote[{\cite[p.150]{lara2013herramientas}}]{Texto}
% --------------------------------------------------------------------------------------------------

% Manipulaciones de TOC

\addtocounter{section}{1}	
\addcontentsline{toc}{section}{\protect\numberline{\thesection}Catalogos}

\addtocounter{subsection}{0}
\addcontentsline{toc}{subsection}{\protect\numberline{\thesubsection}Presion}
\setcounter{subsubsection}{0}

% --------------------------------------------------------------------------------------------------

% plantilla de simplecsv
%{\tiny 
%\centering
%\csvautobooklongtable[separator=comma, respect all, no check column count]{california_housing_train.csv}}

\csvreader[	longtable = cccccccccc,     														% formato de tabla simple
			before table = {\scriptsize \centering},											% codigo antes de \begin{longtable}
			table head=	\caption{Tabla test \label{tab:joder}}\\								% configuracion del header y el foot
			\toprule 1&2&3&4&5&6&7&8&9&10 \\ \midrule \endfirsthead								% como en cualquier otro longtable
			\toprule 1&2&3&4&5&6&7&8&9&10 \\ \midrule \endhead
			\midrule \multicolumn{10}{r}{{continua en la siguiente pagina.}} \\ \bottomrule		% foot
			\endfoot
			\bottomrule \endlastfoot]
			{DataSetP1t=100.csv}
			{}
			{	\csvcoli	& \csvcolii	& \csvcoliii	& \csvcoliv		& 
				\csvcolv	& \csvcolvi	& \csvcolvii	& \csvcolviii	&
				\csvcolix 	& \csvcolx} 			% obligatorio = numero de columnas en romano, 
													% tambien se puede combinar con head names to columns

% --------------------------------------------------------------------------------------------------