% Plantilla para agregar un glosario basico
\newpage
\chapter*{Glosario}
\addcontentsline{toc}{chapter}{\protect Glosario}
	\input{glosario}
% --------------------------------------------------------------------------------------------------

% Plantilla de ecuacion 
\begin{equation}\label{eq:Label}
	\bigtriangleup Rt = R_{o}\alpha\bigtriangleup t 
\end{equation}
% --------------------------------------------------------------------------------------------------

% Plantilla de figura 

\begin{figure}[tb]
	\centering
	\includegraphics[width=\textwidth]{letraA.png} % el archivo debe estar en la carpeta imagenes
	\caption[Titulo de la figura, este texto saldra en la lista de figuras]{\textbf{Titulo en negrita}. Texto descriptivo para la figura} 
	\label{fig:label}
	% [Texto para la lista de figuras]{Texto para el caption}
\end{figure}
% --------------------------------------------------------------------------------------------------

% Plantilla para subfigure
\begin{figure}[tb]
	\centering
	\begin{subfigure}[t]{0.49\textwidth}
		\centering
		\includegraphics[width=\textwidth]{letraA.png} % el archivo debe estar en la carpeta imagenes
		\caption{\textbf{Texto para el caption a}}
		\label{fig:labelA}
	\end{subfigure}
	\hfill
	\begin{subfigure}[t]{0.49\textwidth}
		\centering
		\includegraphics[width=\textwidth]{letraB.png} % el archivo debe estar en la carpeta imagenes
		\caption{\textbf{Texto para el caption b}}
		\label{fig:labelB}
	\end{subfigure}
	
	\caption[Texto para la lista de figuras 1]{\textbf{Titulo en negrita}. Texto para el caption general \label{fig:Label}}
\end{figure}
% --------------------------------------------------------------------------------------------------

% Plantilla para agregar PDF's 
\includepdf[pages={33,19}, % Paginas a agregar {a - b} agrega las paginas de a hasta b
			landscape=true,
			fitpaper,
		%	templatesize={width}{height},
			templatesize={\paperwidth}{\paperheight},
			addtotoc={	33, subsubsection, 3, Indicador de presion, fig:Pindicador, % numero de pagina, seccion,nivel de seccion, texto, label
						19, subsubsection, 3, Accesorios para manómetros, fig:Pacc},
			addtolist={	33, figure, Catalogo de Indicador de presion, fig:Pindicador, % numero de pagina, tipo, texto, label
						19, figure, Catalogo de Accesorios para manómetros, fig:Pacc}]{directorio/"nombre de archivo pdf"}
% --------------------------------------------------------------------------------------------------

% Plantilla para tabla (NO USAR EN FORMATO UNET)
\begin{table}[tb]
	\centering
	\begin{threeparttable}
		\renewcommand{\arraystretch}{1.5} 	% Separacion de las filas
		\setlength{\tabcolsep}{6pt}			% Separacion de columnas
		\caption[Texto para la lista de tablas]{Texto del caption}
		\begin{tabular}{m{4em}m{5em}m{5em}m{7em}m{4em}m{6em}} % m, p y b para texto - c,l y r para numeros y autoexpansion
			\toprule
			 &  &  &  &  &  \\ \midrule
			 &  &  &  &  &  \\
			 &  &  &  &  &  \\ \bottomrule
		\end{tabular}
		\label{tab:Label}
		\begin{tablenotes}[flushleft]
			\item \vspace{7pt}\textbf{Nota.} Texto de la nota de tabla
		\end{tablenotes}
	\end{threeparttable}
\end{table}

% --------------------------------------------------------------------------------------------------

% Plantilla para tabla ajustada al texto (Para formato UNET)
\begin{table}[tb]
	\centering
	\begin{threeparttable}
		\renewcommand{\arraystretch}{1.5} 	% Separacion de las filas
		%\setlength{\tabcolsep}{6pt}			% Separacion de columnas
		\caption[Texto para la lista de tablas]{Texto del caption, el Titulo}
		\begin{tabular*}{\textwidth}{c @{\extracolsep{\fill}} cccccc}
			\toprule
			numeracion & Dato 1 & Dato 2 & Dato 3 & Dato 4 & Dato 5 & Dato 6 \\ \midrule
			1      & 10     &   12   &   14   &   16   &   18   & 30     \\
			2      & 11     &   13   &   15   &   17   &   19   & 31     \\
			3      & 20     &   21   &   22   &   23   &   24   & 32     \\
			4      & 25     &   26   &   27   &   28   &   29   & 33     \\ \bottomrule
		\end{tabular*}
		\label{tab:Label1}
		\begin{tablenotes}[flushleft]
			\item \textbf{Nota.} Texto de la nota de tabla, descripcion de la tabla
		\end{tablenotes}
	\end{threeparttable}
\end{table}

% --------------------------------------------------------------------------------------------------

% Plantilla de sortlist (lista ordenada alfabeticamente)
\begin{sortedlist} 
	\sortitem{ \textbf{Flujo:} Movimiento de un fluido.}
	\sortitem{ \textbf{Flujo:} Movimiento de un fluido.}
\end{sortedlist}
% --------------------------------------------------------------------------------------------------

% Plantilla de footnote
\footnote[1]{} % marcador
\footnotetext[1]{Error combinado del sensor (histéresis, linealidad y repetibilidad)} % texto
% --------------------------------------------------------------------------------------------------

% Plantilla de minted (anexar codigo)
\inputminted[	frame=lines,
				obeytabs=true,
				tabsize=4,
				framesep=5mm,
				bgcolor=white,
				baselinestretch=1.5,
				fontsize=\tiny,]{python}{Ganancias_fuzzy.py}
% --------------------------------------------------------------------------------------------------

% Plantilla para codigo que va en la lista de codigos y con caption
\begin{longlisting}
	\caption{Titulo del codigo}
	\label{code:1}				
	\begin{minted}[escapeinside=||,
		mathescape=true,
		autogobble=true,
		fontsize=\footnotesize,
		obeytabs=true,
		tabsize=4,
		baselinestretch=1]{python}
		Código...
	\end{minted}
\end{longlisting}
% --------------------------------------------------------------------------------------------------

% Plantilla para codigo sin agregar a la lista de codigos y sin caption
\begin{longlisting}			
	\begin{minted}[escapeinside=||,
		mathescape=true,
		autogobble=true,
		fontsize=\footnotesize,
		obeytabs=true,
		tabsize=4,
		baselinestretch=1]{python}
		Código...
	\end{minted}
\end{longlisting}
% --------------------------------------------------------------------------------------------------

% Ejemplos de citas: -------------------------------------------------------------------------------

% Citas parafraseadas

% Cita basada en el autor
bla bla bla \textcite{bibKey} considera que bla bla bla
% Ejemplo: bla bla bla bla \textcite{ogata2003ingenieria} considera que bla bla bla
% Produce: bla bla bla Ogata (2003) considera que bla bla bla

% Cita basada en el texto
bla bla bla y mas bla bla bla \Parencite{bibKey}
% Ejemplo: bla bla bla y mas bla bla bla \Parencite{ogata2003ingenieria}
% Produce: bla bla bla y mas bla bla bla (Ogata, 2003)


% Citas textuales

% Cita textual basada en el autor, menos de 40 palabras:
\textcite{bibKey} afirma que \blockquote[p.150]{Texto}. Continuación

% Ejemplo:
	% bla bla bla bla las das asd asd  aadajlsd aslkdlakjsd as asda bla bla bla. Por algun motivo \textcite{ogata2003ingenieria} afirma que: \blockquote[p.150]{En el análisis presente, se supone que el fluido hidráulico es incompresible y que la fuerza de inercia del pistón de potencia y de la carga es insignificante en comparación con la fuerza hidráulica del pistón de potencia}. Bla bla bla bla las das asd asd  aadajlsd aslkdlakjsd as asda bla bla bla.

% Produce:
%		bla bla bla bla las das asd asd aadajlsd aslkdlakjsd as asda bla bla bla. Por algun
% motivo Ogata (2003) afirma que: "En el análisis presente, se supone que el fluido
% hidráulico es incompresible y	que la fuerza de inercia del pistón de potencia y de la carga
% es insignificante en comparación con la fuerza hidráulica del pistón de potencia" (p.150). Bla
% bla bla bla las das asd asd aadajlsd aslkdlakjsd as asda bla bla bla.



% Cita textual basada en el autor, mas de 40 palabras:
\textcite{bibKey} afirma que: \blockquote[p.150]{Texto.} Continuación

% Ejemplo:
	% bla bla bla bla las das asd asd  aadajlsd aslkdlakjsd as asda bla bla bla. Por algun motivo \textcite{ogata2003ingenieria} afirma que: \blockquote[p.150]{En el análisis presente, se supone que el fluido hidráulico es incompresible y que la fuerza de inercia del pistón de potencia y de la carga es insignificante en comparación con la fuerza hidráulica del pistón de potencia. También se supone que la válvula piloto es una válvula con solape cero y que la velocidad del flujo del aceite es proporcional al desplazamiento de la válvula piloto.} bla bla bla bla las das asd asd  aadajlsd aslkdlakjsd as asda bla bla bla.

% Produce:

%		bla bla bla bla las das asd asd aadajlsd aslkdlakjsd as asda bla bla bla. Por algun
% motivo Ogata (2003) afirma que:
% 		
%		En el análisis presente, se supone que el fluido hidráulico es incompresible y
% 		que la fuerza de inercia del pistón de potencia y de la carga es insignificante
% 		en comparación con la fuerza hidráulica del pistón de potencia. También se supone
% 		que la válvula piloto es una válvula con solape cero y que la velocidad del flujo
% 		del aceite es proporcional al desplazamiento de la válvula piloto. (p.150)

% bla bla bla bla las das asd asd aadajlsd aslkdlakjsd as asda bla bla bla.

% Nota: Blockquote detecta automáticamente si hay mas de 40 palabras, prestar
% atención al punto respecto a menos de 40 palabras



% Cita textual basada en el texto, menos de 40 palabras
bla bla bla \blockquote[{\cite[p.150]{bibKey}}]{Texto}. Continuación

% Ejemplo:
	% bla bla bla bla las das asd asd  aadajlsd aslkdlakjsd as asda bla bla bla y es que \blockquote[{\cite[p.150]{ogata2003ingenieria}}]{En el análisis presente, se supone que el fluido hidráulico es incompresible y que la fuerza de inercia del pistón de potencia y de la carga es insignificante en comparación con la fuerza hidráulica del pistón de potencia}. Bla bla bla bla las das asd asd  aadajlsd aslkdlakjsd as asda bla bla bla.

% Produce:

% 		bla bla bla bla las das asd asd aadajlsd aslkdlakjsd as asda bla bla bla y es que
% "En el análisis presente, se supone que el fluido hidráulico es incompresible y que la
% fuerza de inercia del pistón de potencia y de la carga es insignificante en comparación
% con la fuerza hidráulica del pistón de potencia" (Ogata, 2003, p.150). Bla bla bla bla
% las das asd asd aadajlsd aslkdlakjsd as asda bla bla bla.



% Cita textual basada en el texto, mas de 40 palabras
bla bla bla \blockquote[{\cite[p.150]{bibKey}}]{Texto.} Continuación

% Ejemplo
	% bla bla bla bla las das asd asd  aadajlsd aslkdlakjsd as asda bla bla bla y es lógico que algunas cosas pasan y ademas \blockquote[{\cite[p.150]{ogata2003ingenieria}}]{En el análisis presente, se supone que el fluido hidráulico es incompresible y que la fuerza de inercia del pistón de potencia y de la carga es insignificante en comparación con la fuerza hidráulica del pistón de potencia. También se supone que la válvula piloto es una válvula con solape cero y que la velocidad del flujo del aceite es proporcional al desplazamiento de la válvula piloto.} Bla bla bla bla las das asd asd  aadajlsd aslkdlakjsd as asda bla bla bla.

% Produce:

%		bla bla bla bla las das asd asd  aadajlsd aslkdlakjsd as asda bla bla bla y es lógico que
% algunas cosas pasan y ademas
% 		
%		En el análisis presente, se supone que el fluido hidráulico es incompresible y
% 		que la fuerza de inercia del pistón de potencia y de la carga es insignificante
% 		en comparación con la fuerza hidráulica del pistón de potencia. También se supone
% 		que la válvula piloto es una válvula con solape cero y que la velocidad del flujo
% 		del aceite es proporcional al desplazamiento de la válvula piloto.
%		(Ogata, 2003, p.150)

% bla bla bla bla las das asd asd aadajlsd aslkdlakjsd as asda bla bla bla.
% --------------------------------------------------------------------------------------------------

% Manipulaciones de TOC

\addtocounter{section}{1}	
\addcontentsline{toc}{section}{\protect\numberline{\thesection}Catalogos}

\addtocounter{subsection}{0}
\addcontentsline{toc}{subsection}{\protect\numberline{\thesubsection}Presion}
\setcounter{subsubsection}{0}

% --------------------------------------------------------------------------------------------------

% plantilla de simplecsv
%{\tiny 
%\centering
%\csvautobooklongtable[separator=comma, respect all, no check column count]{california_housing_train.csv}}

\csvreader[	longtable = cccccccccc,     														% formato de tabla simple
			before table = {\scriptsize \centering},											% codigo antes de \begin{longtable}
			table head=	\caption{Tabla test \label{tab:joder}}\\								% configuracion del header y el foot
			\toprule 1&2&3&4&5&6&7&8&9&10 \\ \midrule \endfirsthead								% como en cualquier otro longtable
			\toprule 1&2&3&4&5&6&7&8&9&10 \\ \midrule \endhead
			\midrule \multicolumn{10}{r}{{continua en la siguiente pagina.}} \\ \bottomrule		% foot
			\endfoot
			\bottomrule \endlastfoot]
			{DataSetP1t=100.csv}
			{}
			{	\csvcoli	& \csvcolii	& \csvcoliii	& \csvcoliv		& 
				\csvcolv	& \csvcolvi	& \csvcolvii	& \csvcolviii	&
				\csvcolix 	& \csvcolx} 			% obligatorio = numero de columnas en romano, 
													% tambien se puede combinar con head names to columns

% --------------------------------------------------------------------------------------------------