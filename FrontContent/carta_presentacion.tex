% Para cuando se tenga la carta firmada, scaneada y en formato PDF
% \includepdf[pages=-, fitpaper, templatesize={\paperwidth}{\paperheight},
% 			pagecommand={\thispagestyle{empty}\setcounter{page}{2}}
% 			]{carta_presentacion}

%Generacion de la carta
\begin{titlepage}
\parskip=7.25pt plus 2pt
\setcounter{page}{2}
\begin{flushright}
	San Cristóbal, \today
\end{flushright}

\vspace{1cm}
\vfill

\begin{flushleft}
		\singlespacing
		\setlength{\parskip}{0pt}
		
		\textbf{Señores:}
		
		Miembros de la Comisión del Trabajo de Aplicación Profesional
		
		Departamento de Ingeniería Electrónica
		
\end{flushleft}

\vfill
\begin{spacing}{1.5}
	Yo, !NOMBRE\_AUTOR!, titular de la cédula de identidad No. \mbox{V-x.xxx.xxx}, inscrito en el periodo académico 2019-1, estudiante del !NUMERO\_SEMESTRE! semestre de la carrera de Ingeniería Electrónica, por medio de la presente someto a consideración de la Comisión de Trabajo de Aplicación Profesional de este departamento, la propuesta de Proyecto Especial de Grado titulada \enquote{\textbf{TITULO DE LA TESIS}}, la cual se desarrollará bajo la tutoría del profesor !NOMBRE\_TUTOR!.
	
	Se anexan los recaudos exigidos, de acuerdo a lo dispuesto en la normativa para el Trabajo de Aplicación Profesional de la Universidad Nacional Experimental del Táchira. La fecha estimada de culminación del proyecto es el !FECHA!.
	
	Sin otro particular a que hacer referencia y en espera de su respuesta,
	
	\setlength{\parskip}{20pt} 
	
	\noindent Atentamente,
\end{spacing}

\vfill

\begin{center}
	
	\rule{6cm}{1pt}
	
	\vspace{0.2cm}
	
	\parskip=0pt plus 2pt
    
    \begin{spacing}{1}    
        Nombre del autor
    
        C.I. V-xx.xxx.xxx
    \end{spacing}
\end{center}

\vspace{0.5cm}

\end{titlepage}