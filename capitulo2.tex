\section{Antecedentes}
	
	\Blindtext
	
\section{Bases teóricas}
	
	\blindtext
	
	\subsection{nivel 2}
		
		\blindtext
	
	\subsection{nivel 2}
	
		\blindtext
	
		\subsubsection{nivel 3}
		
			\blindtext
			 
		 \subsubsection{nivel 3}
			 
			 \blindtext
			 
			 \begin{table}[tb]
				\centering
				\begin{threeparttable}
					\renewcommand{\arraystretch}{1.5} 	% Separacion de las filas
					%\setlength{\tabcolsep}{6pt}			% Separacion de columnas
					\caption[Texto para la lista de tablas]{Texto del caption, el Titulo}
					\begin{tabular*}{\textwidth}{c @{\extracolsep{\fill}} cccccc}
						\toprule
						numeracion & Dato 1 & Dato 2 & Dato 3 & Dato 4 & Dato 5 & Dato 6 \\ \midrule
						1      & 10     &   12   &   14   &   16   &   18   & 30     \\
						2      & 11     &   13   &   15   &   17   &   19   & 31     \\
						3      & 20     &   21   &   22   &   23   &   24   & 32     \\
						4      & 25     &   26   &   27   &   28   &   29   & 33     \\ \bottomrule
					\end{tabular*}
					\label{tab:Label1}
					\begin{tablenotes}[flushleft]
						\item \textbf{Nota.} Texto de la nota de tabla, descripcion de la tabla
					\end{tablenotes}
				\end{threeparttable}
			\end{table}
		
		\subsubsection{nivel 3}
			
			\Blindtext

			\blindtext
			
			\begin{figure}[htb]
				\centering
				\includegraphics[width=0.5\textwidth]{letraA.png} % el archivo debe estar en la carpeta imagenes
				\caption[Titulo de la figura, este texto saldra en la lista de figuras]{\textbf{Titulo en negrita}. Texto descriptivo para la figura} 
				\label{fig:letraA}
				% [Texto para la lista de figuras]{Texto para el caption}
			\end{figure}

			\paragraph{nivel 4}

				\blindtext

				\begin{longlisting}
					\caption{Titulo del codigo}
					\label{code:1}				
					\begin{minted}[escapeinside=||,
						mathescape=true,
						autogobble=true,
						fontsize=\footnotesize,
						obeytabs=true,
						tabsize=4,
						baselinestretch=1]{python}
						import control as ctrl
						from matplotlib import pyplot as plt 

						Hs = ctrl.tf([1, 2], [3, 4, 5])
						t, y = ctrl.step_response(Hs)
						plt.plot(t, y, 'r', linestyle=':', linewidth=3)
						plt.grid()
						plt.show()
					\end{minted}
				\end{longlisting}

				\blindtext

				\begin{longlisting}				
					\begin{minted}[escapeinside=||,
						mathescape=true,
						autogobble=true,
						fontsize=\footnotesize,
						obeytabs=true,
						tabsize=4,
						baselinestretch=1]{python}
						# Runge-Kutta para ecuaciones de espacio de estado
						def runge_kutta(t, ss, h, x, u):
							k1 = h * (ss.A * x + ss.B * u)
							k2 = h * (ss.A * (x+k1/2) + ss.B * u)
							k3 = h * (ss.A * (x+k2/2) + ss.B * u)
							k4 = h * (ss.A * (x+k3) + ss.B * u)
							x = x + (1/6)*(k1 + 2*k2 + 2*k3 + k4)
							y = ss.C * x + ss.D * u
							return y, x
					\end{minted}
				\end{longlisting}
	
		
			